\documentclass[11pt]{article}
\usepackage{amsmath,amssymb,amsthm}

\DeclareMathOperator*{\E}{\mathbb{E}}
\let\Pr\relax
\DeclareMathOperator*{\Pr}{\mathbb{P}}

\newcommand{\eps}{\varepsilon}
\newcommand{\inprod}[1]{\left\langle #1 \right\rangle}
\newcommand{\R}{\mathbb{R}}

\newcommand{\handout}[5]{
  \noindent
  \begin{center}
  \framebox{
    \vbox{
      \hbox to 5.78in { {\bf Arena Analytics } \hfill #2 }
      \vspace{4mm}
      \hbox to 5.78in { {\Large \hfill #5  \hfill} }
      \vspace{2mm}
      \hbox to 5.78in { {\em #3 \hfill #4} }
    }
  }
  \end{center}
  \vspace*{4mm}
}

\newcommand{\lecture}[4]{\handout{#1}{#2}{#3}{Escriba: #4}{Sesi\'{o}n #1}}

\newtheorem{theorem}{Teorema}
\newtheorem{corollary}[theorem]{Corolario}
\newtheorem{lemma}[theorem]{Lemma}
\newtheorem{observation}[theorem]{Observación}
\newtheorem{proposition}[theorem]{Proposición}
\newtheorem{definition}[theorem]{Definición}
\newtheorem{claim}[theorem]{Claim}
\newtheorem{fact}[theorem]{Hecho}
\newtheorem{assumption}[theorem]{Suposición}

% 1-inch margins, from fullpage.sty by H.Partl, Version 2, Dec. 15, 1988.
\topmargin 0pt
\advance \topmargin by -\headheight
\advance \topmargin by -\headsep
\textheight 8.9in
\oddsidemargin 0pt
\evensidemargin \oddsidemargin
\marginparwidth 0.5in
\textwidth 6.5in

\parindent 0in
\parskip 1.5ex

\begin{document}

\lecture{2: Descomposición de Fourier, 16/Enero/2019}{Invierno 2019}{Rodolfo Navarrete Pérez}{Rodolfo Navarrete Pérez}

\section{Visión general}

PENDIENDTE





\section{Definiciones y Teoremas}

Sea $T$ un número real positivo, consideramos sólamente funciones reales.

\begin{definition}(Periodicidad)\label{periodicidad}
	Una función es perdiódica de periodo T se cumple la siguiente identidad: 
	\[ f(x+T) = f(x) \quad \forall x \in \mathbb{R} \] 
\end{definition}

\begin{definition}(Frecuencia angular)\label{omega}
	La frecuencia angular de una función periódica de periodo T esta definida por:
	\[ \omega = 2 \pi / T \]
	Adicionalmente, se definen las frecuencias armónicas de T como: 
	\[ \omega_n = n \omega \quad \forall n \in \mathbb{N} \quad n>0 \]
\end{definition}

Es común abusar de la notación y definir $\omega_0, \omega_{-1}, \ldots$ de la misma manera.

\begin{fact}(Ortogonalidad de funciones trigonométricas)\label{ortogonalidad}
	Sean $n$ y $m$ números enteros mayores o iguales a cero tales que $n+m > 0$ (sólamente uno puede ser cero).
	Entonces, para cada $n$ y $m$ se cumplen las siguientes identidades:
	\begin{equation*}
		\begin{split} 
			\int_{-T/2}^{T/2} \sin(n \omega x) \cos(m \omega x) &= 0  \\
			\int_{-T/2}^{T/2} \sin(n \omega x) \sin(m \omega x) &= T/2 \delta_{nm} \\ 
			\int_{-T/2}^{T/2} \cos(n \omega x) \cos(m \omega x) &= T/2 \delta_{nm} \\
		\end{split} 
	\end{equation*}
	Donde $\omega = 2 \pi / T$ y $\delta_{nm}$ es la Delta de Kronecker:
	\[ \delta_{nm} = 
		\begin{cases} 
			1 & \mbox{si } n=m \\
			0 & \mbox{si } n \neq m 
		\end{cases} \]
\end{fact}

\begin{definition}(Producto interno)\label{producto}
	Sean $f$ y $g$ funciones reales periódicas de periodo T, se define su producto interno como: 
	\[ <f,g> = \int_{-T/2}^{T/2} f(x) g(x) dx \]
\end{definition}

La cantidad definida en como ``producto interno" en la definición \ref{producto} cumple los requisitos de ser un producto interno 
del espacio vectorial formado por las funciones periódicas de periodo $T$ (conmutatividad, bilinealidad, positivo definido). 
Las demostraciones de que las funciones periódicas forman un espacio vectorial 
y de que el producto definido por su integral es en efecto un producto interno 
no se darán en estas notas.

\section{Descomposición de Fourier}

\begin{equation}
	dsa
\end{equation}


\subsection{Blah blah blah}
Here is a subsection.

\subsubsection{Blah blah blah}
Here is a subsubsection. You can use these as well.

\subsection{Using Boldface}
Make sure to use \textbf{lots} of {\bf boldface}.

\paragraph{Question:}
How would you use boldface?

\paragraph{Example:}
This is an example showing how to use boldface to 
help organize your lectures.


\paragraph{Some Formatting.}
Here is some formatting that you can use in your notes:
\begin{itemize}
\item {\em Item One} -- This is the first item.
\item {\em Item Two} -- This is the second item.
\item \dots and here are other items.
\end{itemize}

If you need to number things, you can use this style:
\begin{enumerate}
\item {\em Item One} -- Again, this is the first item.
\item {\em Item Two} -- Again, this is the second item.
\item \dots and here are other items.
\end{enumerate}

\paragraph{Bibliography.}
Please give real bibliographical citations for the papers that we
mention in class. See below for how to include a bibliography section.
If you use BibTeX, integrate the .bbl file into your .tex
source. You should reference papers like this: ``The tug of war sketch
originates in a paper by Alon, Matias and Szegedy \cite{AlonMS99}.''
In general, the name of the authors should appear in text at most once 
(for the first citation); further citations look like: ``Our proof follows 
that of \cite{AlonMS99}''.

Take a look at previous lectures (TeX files are available) to see the
details. A excellent source for bibliographical citations is
DBLP. Just Google DBLP and an author's name.


\bibliographystyle{alpha}

\begin{thebibliography}{42}

\bibitem{AlonMS99}
Noga~Alon, Yossi~Matias, Mario~Szegedy.
\newblock The Space Complexity of Approximating the Frequency Moments.
\newblock {\em J. Comput. Syst. Sci.}, 58(1):137--147, 1999.

\end{thebibliography}

\end{document}
